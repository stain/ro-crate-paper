\markdownRendererHeadingOne{RO-Crate}\markdownRendererInterblockSeparator
{}RO-Crate provides a lightweight approach to packaging research artefacts with their metadata. To illustrate this, let us imagine a research paper reporting on the sequence analysis of proteins obtained from an experiment on mice. The sequence output files, sequence analysis code, resulting data and reports summarising statistical measures or outputs are all important and inter-related research outputs, and consequently would ideally all be co-located in a directory and accompanied with their corresponding metadata. In reality, some of the artefacts (e.g. data or software) will be recorded as external references, not necessarily captured in a FAIR way. This directory, along with the relationships between its constituent digital artefacts, is what the RO-Crate model aims to represent, linking together all the elements pertaining to an experiment and required for its reproducibility. \markdownRendererInterblockSeparator
{}The question then arises as to how the directory with all this material should be packaged in a manner that is accessible and usable by others. By usable we mean not just readable by humans but programmatically accessible. A de facto approach to sharing collections of resources is through compressed archives (e.g. a zip file). This solves the problem of “packaging”, but it does not guarantee downstream access to all artefacts in a programmatic fashion, or the role of each file in that particular research. This leads to the need for explicit metadata about the contents of the folder, describing each and linking them together.\markdownRendererInterblockSeparator
{}Examples of metadata descriptions across a \markdownRendererLink{wide range of domains}{https://rdamsc.bath.ac.uk/scheme-index}{https://rdamsc.bath.ac.uk/scheme-index}{} abound within the literature, both in research data management (?cite) and within library and information systems (?cite). However, many of these approaches require knowledge of metadata schemas, particular annotation systems, or the use of obscure or complex software stacks. Indeed, particularly within research, these requirements have led to a lack of adoption and growing frustration with current tooling and specifications \markdownRendererCite{1}+{}{}{neylon_blog_post_2017}.\markdownRendererInterblockSeparator
{}RO-Crate seeks to address this complexity by:\markdownRendererInterblockSeparator
{}\markdownRendererOlBeginTight
\markdownRendererOlItemWithNumber{1}being easy to understand and conceptually simple;\markdownRendererOlItemEnd 
\markdownRendererOlItemWithNumber{2}providing a strong and opinionated guide regarding current best practices;\markdownRendererOlItemEnd 
\markdownRendererOlItemWithNumber{3}adopting de-facto standards that are widely used on the Web.\markdownRendererOlItemEnd 
\markdownRendererOlEndTight \markdownRendererInterblockSeparator
{}In the following sections we show how the RO-Crate specification and ecosystem achieves these goals, which concur in forming our definition of “lightweight”. \relax