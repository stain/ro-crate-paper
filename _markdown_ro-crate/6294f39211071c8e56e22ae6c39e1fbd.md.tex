\markdownRendererHeadingTwo{Bioinformatics workflows}\markdownRendererInterblockSeparator
{}\markdownRendererLink{WorkflowHub.eu}{https://workflowhub.eu/}{https://workflowhub.eu/}{} is a European cross-domain registry of computational workflows, supported by European Open Science Cloud projects, e.g. \markdownRendererLink{EOSC-Life}{https://www.eosc-life.eu/}{https://www.eosc-life.eu/}{}, and research infrastructures including the pan-European bioinformatics network \markdownRendererLink{ELIXIR}{https://elixir-europe.org/}{https://elixir-europe.org/}{} \markdownRendererCite{1}+{}{}{doi:10.1016/j.tibtech.2012.02.002}. As part of promoting workflows as reusable tools, WorkflowHub includes documentation and high-level rendering of the workflow structure independent of its native workflow definition format. The rationale is that a domain scientist can browse all relevant workflows for their domain, before narrowing down their workflow engine requirements. As such, the WorkflowHub is intended largely as a registry of workflows already deposited in repositories specific to particular workflow languages and domains, such as UseGalaxy.eu \markdownRendererCite{1}+{}{}{doi:10.1371/journal.ppat.1008643} and Nextflow nf-core \markdownRendererCite{1}+{}{}{doi:10.1038/s41587-020-0439-x}. \markdownRendererInterblockSeparator
{}We here describe three different RO-Crate profiles developed for use with WorkflowHub.\markdownRendererInterblockSeparator
{}\markdownRendererHeadingThree{Profile for describing workflows}\markdownRendererInterblockSeparator
{}Being cross-domain, WorkflowHub has to cater for many different workflow systems. Many of these, for instance Nextflow \markdownRendererCite{1}+{}{}{doi:10.1038/nbt.3820} and Snakemake \markdownRendererCite{1}+{}{}{doi:10.1093/bioinformatics/bts480}, by virtue of their script-like nature, reference multiple neighbouring files typically maintained in a GitHub repository. This calls for a data exchange method that allows keeping related files together. WorkflowHub has tackled this problem by adopting RO-Crate as the packaging mechanism \markdownRendererCite{1}+{}{}{doi:10.5281/zenodo.4705078}, typing and annotating the constituent files of a workflow and — crucially — marking up the workflow language, as many workflow engines use common file extensions like \markdownRendererCodeSpan{*.xml} and \markdownRendererCodeSpan{*.json}. Workflows are further described with authors, licence, diagram previews and a listing of their inputs and outputs. RO-Crates can thus be used for interoperable deposition of workflows to WorkflowHub, but are also used as an archive for downloading workflows, embedding metadata registered with the WorkflowHub entry and translated workflow files such as abstract Common Workflow Language (CWL) \markdownRendererCite{1}+{}{}{arxiv:2105.07028} definitions and diagrams \markdownRendererCite{1}+{}{}{doi:10.5281/zenodo.4605654}. \markdownRendererInterblockSeparator
{}RO-Crate acts therefore as an interoperability layer between registries, repositories and users in WorkflowHub. The iterative development between WorkflowHub developers and the RO-Crate community heavily informed the creation of the Bioschemas \markdownRendererCite{1}+{}{}{bioschemas_2017} profile for \markdownRendererLink{Computational Workflows}{https://bioschemas.org/profiles/ComputationalWorkflow/1.0-RELEASE/}{https://bioschemas.org/profiles/ComputationalWorkflow/1.0-RELEASE/}{}, which again informed the \markdownRendererLink{RO-Crate 1.1 specification on workflows}{https://www.researchobject.org/ro-crate/1.1/workflows.html}{https://www.researchobject.org/ro-crate/1.1/workflows.html}{} and led to the RO-Crate Python library \markdownRendererCite{1}+{}{}{ro-crate-py} and WorkflowHub’s \markdownRendererLink{\markdownRendererStrongEmphasis{Workflow RO-Crate profile}}{https://about.workflowhub.eu/Workflow-RO-Crate/}{https://about.workflowhub.eu/Workflow-RO-Crate/}{}, which, in a similar fashion to RO-Crate itself, recommends which workflow resources and descriptions are required. This co-development across project boundaries exemplifies the drive for simplicity and for establishing best practices.\markdownRendererInterblockSeparator
{}\markdownRendererHeadingThree{Profile for recording workflow runs}\markdownRendererInterblockSeparator
{}While RO-Crates in WorkflowHub so far have been focused on workflows that are ready to be run, development of WorkflowHub are now creating a \markdownRendererStrongEmphasis{Workflow Run RO-Crate profile} for the purposes of benchmarking, testing and executing workflows. As such, RO-Crate serves as a container of both a \markdownRendererEmphasis{workflow definition} that may be executed and of a particular \markdownRendererEmphasis{workflow execution with test results}. This profile is a continuation of our previous work with capturing workflow provenance in a Research Object in CWLProv \markdownRendererCite{1}+{}{}{doi:10.1093/gigascience/giz095} and TavernaPROV \markdownRendererCite{1}+{}{}{doi:10.5281/zenodo.51314}. In both cases, we used the PROV Ontology \markdownRendererCite{1}+{}{}{PROVO}, including details of every task execution with all the intermediate data, which required significant workflow engine integration\markdownRendererFootnote{CWLProv and TavernaProv predate RO-Crate, but use RO-Bundle\markdownRendererCite{1}+{}{}{doi:10.5281/zenodo.12586}, a similar Research Object packaging method with JSON-LD metadata. }. To simplify from that approach, for this Workflow Run RO-Crate profile we will use a higher level \markdownRendererLink{schema.org provenance}{https://www.researchobject.org/ro-crate/1.1/provenance.html#software-used-to-create-files}{https://www.researchobject.org/ro-crate/1.1/provenance.html#software-used-to-create-files}{} for the input/output boundary of the overall workflow execution. This \markdownRendererEmphasis{Level 1 workflow provenance} \markdownRendererCite{1}+{}{}{doi:10.1093/gigascience/giz095} can be expressed generally across workflow languages with minimal engine changes, with the option of more detailed provenance traces as separate PROV resources in the RO-Crate. \markdownRendererInterblockSeparator
{}WorkflowHub has recently enabled minting of Digital Object Identifiers (DOIs), a PID commonly used for scholarly artefacts, for registered workflows, e.g. \markdownRendererCodeSpan{10.48546/workflowhub.workflow.56.1} \markdownRendererCite{1}+{}{}{doi:10.48546/workflowhub.workflow.56.1}, lowering the barrier for citing workflows as computational methods along with their FAIR metadata – captured within an RO-Crate. While it is not an aim for WorkflowHub to be a repository of workflow runs and their data, RO-Crates of \markdownRendererEmphasis{exemplar workflow runs} serve as useful workflow documentation, as well as being an exchange mechanism that preserve FAIR metadata in a diverse workflow execution environment.\markdownRendererInterblockSeparator
{}\markdownRendererHeadingThree{Profile for testing workflows}\markdownRendererInterblockSeparator
{}The value of computational workflows, however, is potentially undermined by the "collapse" over time of the software and services they depend upon: for instance, software dependencies can change in a non-backwards-compatible manner, or active maintenance may cease; an external resource, such as a reference index or a database query service, could shift to a different URL or modify its access protocol; or the workflow itself may develop hard-to-find bugs as it is updated. This can take a big toll on the workflow's reusability and on the reproducibility of any processes it evokes.\markdownRendererInterblockSeparator
{}For this reason, WorkflowHub is complemented by a monitoring and testing service called LifeMonitor\markdownRendererCite{1}+{}{}{about-lifemonitor}, also supported by EOSC-Life. LifeMonitor's main goal is to assist in the creation, periodic execution and monitoring of workflow tests, enabling the early detection of software collapse in order to minimise its detrimental effects. The communication of metadata related to workflow testing is achieved through the adoption of a \markdownRendererLink{\markdownRendererStrongEmphasis{Workflow Testing RO-Crate profile}}{https://crs4.github.io/life_monitor/workflow_testing_ro_crate}{https://crs4.github.io/life_monitor/workflow_testing_ro_crate}{} stacked on top of the \markdownRendererEmphasis{Workflow RO-Crate} profile. This further specialisation of Workflow RO-Crate allows to specify additional testing-related entities (test suites, instances, services, etc.), leveraging \markdownRendererLink{RO-Crate's extension mechanism}{https://www.researchobject.org/ro-crate/1.1/appendix/jsonld.html#extending-ro-crate}{https://www.researchobject.org/ro-crate/1.1/appendix/jsonld.html#extending-ro-crate}{} through the addition of terms from custom namespaces.\markdownRendererInterblockSeparator
{}In addition to showcasing RO-Crate's extensibility, the testing profile is an example of the format's flexibility and adaptability to the different needs of the research community. Though ultimately related to a computational workflow, in fact, most of the testing-specific entities are more about describing a protocol for interacting with a monitoring service than a set of research outputs and its associated metadata. Indeed, one of LifeMonitor's main functionalities is monitoring and reporting on test suites running on existing Continuous Integration (CI) services, which is described in terms of service URLs and job identifiers in the testing profile. In principle, in this context, data could disappear altogether, leading to an RO-Crate consisting entirely of contextual entities. Such an RO-Crate acts more as an exchange format for communication between services (WorkflowHub and LifeMonitor) than as an aggregator for research data and metadata, providing a good example of the format's high versatility.\relax