\markdownRendererHeadingTwo{Digital Humanities: Cultural Heritage}\markdownRendererInterblockSeparator
{}\markdownRendererLink{PARADISEC}{https://www.paradisec.org.au/}{https://www.paradisec.org.au/}{} (the Pacific And Regional Archive for Digital Sources in Endangered Cultures) maintains a repository of more than 500,000 files documenting endangered languages across more than 16,000 items, collected over many years by researchers interviewing and recording native speakers across the region. As a proposed update of the 18 year old infrastructure, the \markdownRendererLink{Modern PARADISEC demonstrator}{https://mod.paradisec.org.au/}{https://mod.paradisec.org.au/}{} has been \markdownRendererLink{developed}{https://arkisto-platform.github.io/case-studies/paradisec/}{https://arkisto-platform.github.io/case-studies/paradisec/}{} to also help digitally preserve these artefacts using the \markdownRendererLink{Oxford Common File Layout}{https://ocfl.io/1.0/spec/}{https://ocfl.io/1.0/spec/}{} (OCFL) for file consistency and RO-Crate for structuring and capturing the metadata of each item. The existing PARADISEC data collection has been ported and captured as RO-Crates. A web portal then exposes the repository and its entries by indexing the RO-Crate metadata files using Elasticsearch as a “NoSQL” object database, presenting a domain-specific view of the items - the RO-Crate is “hidden” and does not change the user interface.\markdownRendererInterblockSeparator
{}This use case takes advantage of several RO-Crate features and principles. Firstly, the transcribed metadata are now independent of the PARADISEC platform and can be archived, preserved and processed in its own right, using Schema.org vocabularies augmented with PARADISEC-specific terms. The lightweight infrastructure with RO-Crate as the holder of itemised metadata in regular files (organised using OCFL\markdownRendererCite{1}+{}{}{ocfl_2020}, with checksums for integrity checking and versioning) also gives flexibility for future developments and maintenance; for example, potentially using Linked Data software such as a graph database, queried using SPARQL triple patterns across RO-Crates, or a “last resort” fallback to the generic RO-Crate HTML preview \markdownRendererCite{1}+{}{}{ro-crate-html-js}, which can be hosted as static files by any web server, in line with the approach taken by the Endings Project\markdownRendererFootnote{The Endings Project \url{https://endings.uvic.ca/} is a five-year project funded by the Social Sciences and Humanities Research Council (SSHRC) that is creating tools, principles, policies and recommendations for digital scholarship practitioners to create accessible, stable, long-lasting resources in the humanities.}.\relax