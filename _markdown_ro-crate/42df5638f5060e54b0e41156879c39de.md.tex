\markdownRendererHeadingOne{Conclusion}\markdownRendererInterblockSeparator
{}RO-Crate provides a lightweight approach to packaging digital research artefacts with structured metadata, assisting developers and researchers to produce and consume FAIR data archives of their ROs, including PIDs, context and provenance of research artefacts. Aggregated data may be large and distributed, or located in regular folders on a file system. \markdownRendererInterblockSeparator
{}As a set of best practice recommendations, developed by an open and broad community, RO-Crate shows how to use "just enough" Linked Data standards in a consistent way, with structured metadata using a rich base vocabulary that can cover general-purpose contextual relations, whilst retaining extensibility to domain- and application-specific uses. \markdownRendererInterblockSeparator
{}The adoption of simple web technologies in the RO-Crate specification has helped a rapid development of a wide variety of supporting open source tools and libraries. RO-Crate fits into the larger landscape of open scholarly communication and FAIR Digital Object infrastructure, and can be integrated into data repository platforms. RO-Crate can be applied as a data/metadata exchange mechanism, assist in long-term archival preservation of metadata and data, or simply used at small-scale by individual researchers. Thanks to its strong community support, new and improved profiles and tools are continuously added to the RO-Crate tooling landscape, making it easier for adopters to find examples and support for their own use case.\relax