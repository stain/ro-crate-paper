\markdownRendererHeadingTwo{RO-Crate Community}\markdownRendererInterblockSeparator
{}The RO-Crate conceptual model, implementation and best practices are developed by a growing community of researchers, developers and publishers. The RO-Crate community is a key aspect of its effectiveness in making research artefacts FAIR. Fundamentally, the community provides the overall context of the implementation and model and ensures its interoperability. \markdownRendererInterblockSeparator
{}The RO-Crate community consists of:\markdownRendererInterblockSeparator
{}\markdownRendererOlBeginTight
\markdownRendererOlItemWithNumber{1}a diverse set of people representing a variety of stakeholders;\markdownRendererOlItemEnd 
\markdownRendererOlItemWithNumber{2}a set of collective norms;\markdownRendererOlItemEnd 
\markdownRendererOlItemWithNumber{3}an open platform that facilitates communication (GitHub, Google Docs, monthly teleconferences).\markdownRendererOlItemEnd 
\markdownRendererOlEndTight \markdownRendererInterblockSeparator
{}\markdownRendererHeadingThree{People}\markdownRendererInterblockSeparator
{}The initial concept of RO-Crate was formed at the first Workshop on Research Objects (\markdownRendererLink{RO2018}{https://www.researchobject.org/ro2018/}{https://www.researchobject.org/ro2018/}{}), held as part of the IEEE conference on eScience. This workshop followed up on considerations made at a \markdownRendererLink{Research Data Alliance (RDA) meeting on Research Data Packaging}{https://rd-alliance.org/approaches-research-data-packaging-rda-11th-plenary-bof-meeting}{https://rd-alliance.org/approaches-research-data-packaging-rda-11th-plenary-bof-meeting}{} that found similar goals across multiple data packaging efforts \markdownRendererCite{1}+{}{}{doi:10.5281/zenodo.3250687}: simplicity, structured metadata and the use of JSON-LD.\markdownRendererInterblockSeparator
{}An important outcome of discussions that took place at RO2018 was the conclusion that the original Wf4Ever Research Object ontologies \markdownRendererCite{1}+{}{}{doi:10.1016/j.websem.2015.01.003}, in principle sufficient for packaging research artefacts with rich descriptions, were, in practice, considered inaccessible for regular programmers (e.g. web developers) and in danger of being incomprehensible for domain scientists due to their reliance on Semantic Web technologies and other ontologies.\markdownRendererInterblockSeparator
{}DataCrate \markdownRendererCite{1}+{}{}{doi:10.5281/zenodo.1445817} was presented at RO2018 as a promising lightweight alternative approach, and an agreement was made by a group of volunteers to attempt building “RO Lite” as a combination of DataCrate's implementation and Research Object's principles.\markdownRendererInterblockSeparator
{}This group, originally made up of library and Semantic Web experts, has subsequently grown to include domain scientists, developers, publishers and more. This perspective of multiple views led to the specification being used in a variety of domains, from bioinformatics and regulatory submissions to humanities and cultural heritage preservation. \markdownRendererInterblockSeparator
{}The RO-Crate community is strongly engaged with the European-wide biology/bioinformatics collaborative e-Infrastructure, ELIXIR, \markdownRendererCite{1}+{}{}{doi:10.1016/j.tibtech.2012.02.002}, along with European Open Science Cloud (EOSC) projects including EOSC-Life and FAIRplus. RO-Crate has also established collaborations with Bioschemas, GA4GH, OpenAIRE and multiple H2020 projects.\markdownRendererInterblockSeparator
{}A key set of stakeholders are developers; the RO-Crate community has made a point of attracting developers who can implement the specifications but, importantly, keeps “developer user experience” in mind. This means that the specifications are straightforward to implement and thus do not require expertise in technologies that are not widely deployed. \markdownRendererInterblockSeparator
{}This notion of catering to “developer user experience” is an example of the set of norms that have developed and now define the community. \markdownRendererInterblockSeparator
{}\markdownRendererHeadingThree{Norms}\markdownRendererInterblockSeparator
{}The RO-Crate community is driven by conventions or notions that are prevalent within it but not formalised. Here, we distil what we as authors believe are the critical set of norms that have facilitated the development of RO-Crate and contributed to the ability for RO-Crate research packages to be FAIR. This is not to say that there are no other norms within the community or that everyone in the community holds these uniformly. Instead, what we emphasise is that these norms are helpful and also shaped by community practices. \markdownRendererInterblockSeparator
{}\markdownRendererOlBeginTight
\markdownRendererOlItemWithNumber{1}Simplicity\markdownRendererOlItemEnd 
\markdownRendererOlItemWithNumber{2}Developer friendliness\markdownRendererOlItemEnd 
\markdownRendererOlItemWithNumber{3}Focus on examples and best practices rather than rigorous specification\markdownRendererOlItemEnd 
\markdownRendererOlItemWithNumber{4}Reuse “just enough” Web standards\markdownRendererOlItemEnd 
\markdownRendererOlEndTight \markdownRendererInterblockSeparator
{}A core norm of RO-Crate is that of \markdownRendererStrongEmphasis{simplicity}, which sets the scene for how we guide developers to structure metadata with RO-Crate. We focus mainly on documenting simple approaches to the most common use cases, such as authors having an affiliation. This norm also influences our take on \markdownRendererStrongEmphasis{developer friendliness}; for instance, we are using the Web-native JSON format, allowing only a few of JSON-LD's flexible Linked Data features. Moreover, the RO-Crate documentation is largely built up by \markdownRendererStrongEmphasis{examples} showcasing \markdownRendererStrongEmphasis{best practices}, rather than rigorous specifications. In a sense, we are allowed to do this by building on existing \markdownRendererStrongEmphasis{Web standards} that themselves are defined rigorously, which we utilise \markdownRendererEmphasis{“\markdownRendererStrongEmphasis{just enough}”} in order to benefit from the advantages of Linked Data (e.g. extensions by namespaced vocabularies), without imposing too many developer choices or uncertainties (e.g. having to choose between the many RDF syntaxes). \markdownRendererInterblockSeparator
{}While the above norms alone could easily lead to the creation of “yet another” JSON format, we keep the goal of \markdownRendererStrongEmphasis{FAIR interoperability} of the captured metadata, and therefore follow closely FAIR best practices and current developments such as data citations, PIDs, open repositories and recommendations for sharing research outputs and software.\markdownRendererInterblockSeparator
{}\markdownRendererHeadingThree{Open Platforms}\markdownRendererInterblockSeparator
{}The critical infrastructure that enables the community around RO-Crate is the use of open development platforms. This underpins the importance of open community access to supporting FAIR. Specifically, it is difficult to build and consume FAIR research artefacts without being able to access the specifications, understand how they are developed, know about any potential implementation issues, and discuss usage to evolve best practices. \markdownRendererInterblockSeparator
{}It was seen as more important to document real-life examples and best practices rather than developing a rigorous specification. At the same time, we agreed to be opinionated on the syntactic form to reduce the jungle of implementation choices; we wanted to keep the important aspects of Linked Data to adhere to the FAIR principles while retaining the option of combining and extending the structured metadata using the existing Semantic Web stack, not just build “yet another” standalone JSON format. \markdownRendererInterblockSeparator
{}Further work during 2019 started adapting the DataCrate documentation through a more collaborative and exploratory \markdownRendererEmphasis{RO-Lite} phase, initially using Google Docs for review and discussion, then moving to GitHub as a collaboration space for developing what is now the RO-Crate specification, \markdownRendererLink{maintained as Markdown}{https://github.com/researchobject/ro-crate/}{https://github.com/researchobject/ro-crate/}{} in GitHub Pages and published through Zenodo. \markdownRendererInterblockSeparator
{}In addition to the typical Open Source-style development with GitHub issues and pull requests, the RO-Crate Community now has two regular monthly calls, a Slack channel and a mailing list for coordinating the project, and many of its participants collaborate on RO-Crate at multiple conferences and coding events such as the ELIXIR BioHackathon. The community is jointly developing the RO-Crate specification and Open Source tools, as well as providing support and considering new use cases. The \markdownRendererLink{RO-Crate Community}{https://www.researchobject.org/ro-crate/community}{https://www.researchobject.org/ro-crate/community}{} is open for anyone to join, to equally participate under a code of conduct, and currently has more than 40 members. \relax