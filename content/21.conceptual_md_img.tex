## Conceptual Definition

A key premise of RO-Crate is the existence of a wide variety of resources on the Web that can help describe research. As such, RO-Crate relies on the Linked Data principles [@doi:10.2200/S00334ED1V01Y201102WBE001]. Figure {@fig:conceptual} \ref{fig:conceptual} shows the main conceptual elements involved in an RO-Crate: an RO-Metadata File (top) describes the research object using structured metadata including external references, coupled with the contained artefacts (bottom) bundled and described by the RO-Crate.

![Conceptual RO-Crate Overview](../content/images/ro-crate-overview.pdf "\textbf{Conceptual overview of RO-Crate}. A \emph{Persistent Identifier} (PID) \cite{doi:10.1371/journal.pbio.2001414} points to a \emph{Research Object} (RO), which may be archived using different packaging approaches like BagIt \cite{doi:10.17487/rfc8493}, OCFL \cite{ocfl_2020}, git or ZIP. The RO is described within a \emph{RO-Crate Metadata File}, providing identifiers for \emph{authors} using ORCID, \emph{organisations} using [ROR](https://ror.org/) and licences such as Creative Commons using SPDX identifiers. The \emph{RO-Crate content} is further described with additional metadata. Data can be embedded files and directories, as well as links to external web resources, PIDs and nested RO-Crates."){#fig:conceptual width="90%"}

### Linked Data as a foundation

The **Linked Data** principles [@doi:10.4018/978-1-60960-593-3.ch008] (use of IRIs[^1] to identify resources (i.e. artefacts), resolvable via HTTP, enriched with metadata and linked to each other) are core to RO-Crate; therefore IRIs are used to identify an RO-Crate, its constituent parts and metadata descriptions, and the properties and classes used in the metadata. 

RO-Crates are _self-described_; and follow the Linked Data principles to describe all of their resources in both human and machine readable manner.  Hence, resources are identified using global identifiers where possible; and relationships between two resources are defined with links.

The foundation of Linked Data and shared vocabularies also means multiple RO-Crates and other Linked Data resources can be indexed, combined, queried or transformed using existing Semantic Web technologies such as [SPARQL](https://www.w3.org/TR/sparql11-overview) and knowledge graph triple stores.

Even though an RO-Crate is not required to be published on the Web, this use of mature web technologies means its developers and consumers are not restricted to the Research Object aspects that have already been specified by the RO-Crate community, but can extend and integrate in multiple standardized ways. 


### RO-Crate is a self-described container

An [RO-Crate is defined](https://www.researchobject.org/ro-crate/1.1/structure.html#ro-crate-metadata-file-ro-crate-metadatajson) as a self-described **Root Data Entity** that describes and contains _data entities_, which are further described using _contextual entities_.  A **data entity** is either a _file_ (i.e. a set of bytes stored on disk somewhere) or a _directory_ (i.e. dataset of named files and other directories). A file does not need to be stored inside the RO-Crate root, it can be referenced via a PID/IRI. A **contextual entity** exists outside the information system (e.g. a Person, a workflow language) and is defined by its metadata. The representation of a **data entity** as a set of bytes makes it possible to store a variety of research artefacts including not only data but also, for instance, software and text.

Any particular IRI might appear as a contextual entity in one RO-Crate and as a data entity in another; their distinction lies in the fact that data entities can be considered to be _contained_ or captured by the RO-Crate (_RO Content_ in {@fig:conceptual}), while contextual entities mainly _explain_ the RO-Crate, although this distinction is not a formal requirement.

The Root Data Entity is a directory, the RO-Crate root, identified by the presence of the **RO-Crate Metadata File** (`ro-crate-metadata.json`) (Figure \ref{fig:conceptual} top). This is a JSON-LD file that describes the RO-Crate, its content and related metadata using Linked Data. JSON-LD is a W3C standard RDF serialisation that has become popular as it is easy to read by humans while also offers some advantages for data exchange on the Internet. JSON-LD is the preferred and widely supported format by RO-Crate tools and community.

The minimal [requirements for the root data entity metadata](https://www.researchobject.org/ro-crate/1.1/root-data-entity.html#direct-properties-of-the-root-data-entity) are `name`, `description` and `datePublished`, as well as a contextual entity identifying its `license` — additional metadata are frequently added to entities depending on the purpose of the particular RO-Crate.

RO-Crate can be stored, transferred or published in multiple ways, e.g. BagIt [@doi:10.17487/rfc8493], Oxford Common File Layout [@ocfl_2020] (OCFL), downloadable ZIP archives in Zenodo or through dedicated online repositories, as well as published directly on the Web, e.g. using [GitHub Pages](https://pages.github.com/). Combined with Linked Data identifiers, this caters for a diverse set of storage and access requirements across different scientific domains, from metagenomics workflows producing hundreds of gigabytes of genome data to cultural heritage records with access restrictions for personally identifiable data. Specific [RO-Crate profiles](https://www.researchobject.org/ro-crate/1.2-DRAFT/profiles.html) may constrain serialization and publication expectations, and require additional contextual types and properties.

### Data Entities are described using Contextual Entities

RO-Crate distinguishes between [data and contextual entities](https://www.researchobject.org/ro-crate/1.1/contextual-entities.html#contextual-vs-data-entities) in a similar way to HTTP terminology's early attempt to separate _information_ (data) and _non-information_ (contextual) resources [@httprange14]. Data entities are usually files and directories located by relative IRI references within the RO-Crate Root, but they can also be Web resources or restricted data identified with absolute IRIs.

As both types of entities are identified by IRIs, their distinction is allowed to be blurry; data entities can be located anywhere and be complex, while contextual entities can have a Web presence beyond their description inside the RO-Crate. For instance `https://orcid.org/0000-0002-1825-0097` is primarily an identifier for a person, but secondarily it is also a web page and a way to refer to their academic work. It follows that an RO-Crate should include a contextual entity that describes that person. 

Any particular IRI might appear as a contextual entity in one RO-Crate and as a data entity in another; their distinction lies in the fact that data entities can be considered to be _contained_ or captured by the RO-Crate, while contextual entities mainly _explain_ the RO-Crate and its entities. 

Figure \ref{fig:uml} shows a UML view of RO-Crate, highlighting the different types of data entities and contextual entities that can be aggregated and related. While an RO-Crate would usually contain one or more data entities (`hasPart`), it may also be a pure aggregation of contextual entities (`mentions`).

![RO-Crate UML](../content/images/ro-crate-uml.pdf "\textbf{UML model view of RO-Crate.} The \emph{RO-Crate Metadata File} conforms to a version of the specification; and contains a JSON-LD graph that describes the entities that make up the RO-Crate. The \emph{RO-Crate Root Data Entity} represent the Research Object as a dataset. The RO-Crate aggregates \emph{data entities} (\texttt{hasPart}) which are further described using \emph{contextual entities} (which may include aggregated and non-aggregated data entities). Multiple types and relations from Schema.org allow annotations to be more specific, including figures, nested datasets, computational workflows, people, organisations, instruments and places. Contextual entities not otherwise cross-referenced from other entities' properties (\emph{describes}) can be grouped under the root entity (\texttt{mentions})."){#fig:uml width="90%"}



### Guide through Recommended Practices

RO-Crate as a specification aims to build a set of recommended practices on how to practically apply existing standards in a common way to describe research outputs and their provenance, without having to learn each of the underlying technologies in detail.

As such, the [RO-Crate 1.1](https://w3id.org/ro/crate/1.1) specification [@doi:10.5281/zenodo.4541002] can be seen as an opinionated and example-driven guide to writing [Schema.org](https://schema.org/) [@doi:10.1145/2857274.2857276]) metadata as JSON-LD [@sporny_2014], which leaves it open for implementers to include additional metadata using other Schema.org types and properties, or even additional Linked Data vocabularies/ontologies or their own ad-hoc terms.

However the primary purpose of the RO-Crate specification is to assist developers in leveraging Linked Data principles for the focused purpose of describing Research Objects in a structured language, while reducing the steep learning curve otherwise involved in Semantic Web adaptation, like ontology development and selection, identifiers, namespaces, and RDF serialization choices.

### Ensuring Simplicity

One aim of RO-Crate is to be conceptually simple. This simplicity has been repeatedly checked and confirmed through an informal community review process. For instance, in the discussion on supporting [ad-hoc vocabularies](https://github.com/ResearchObject/ro-crate/issues/71) in RO-Crate, the community explored potential Linked Data solutions. The conventional wisdom in [RDF best practice](https://www.w3.org/TR/swbp-vocab-pub/) is to establish a vocabulary with a new URI namespace, formalised using [RDF Schema](http://www.w3.org/TR/2014/REC-rdf-schema-20140225/) or [OWL](http://www.w3.org/TR/2012/REC-owl2-overview-20121211/) ontologies.
However, this may seem excessive learning curve for non-experts in semantic knowledge representation, and the RO-Crate community instead agreed on a dual lightweight approach: (ⅰ) [Document](https://www.researchobject.org/ro-crate/1.1/appendix/jsonld.html#adding-new-or-ad-hoc-vocabulary-terms) how projects with their own web-presence can make a pure HTML-based vocabulary, and (ⅱ) provide a community-wide PID namespace under <https://w3id.org/ro/terms/> that redirect to simple CSV files [maintained in GitHub](https://github.com/ResearchObject/ro-terms). 

To further verify this idea, we have formalised the RO-Crate definition (see _Appendix A_). An important result of this exercise is that the underlying data structure of RO-Crate, although conceptually a graph, is represented as a depth-limited tree. This formalisation also emphasises the _boundedness_ of the structure; namely, the fact that elements are specifically identified as being either semantically _contained_ by the RO-Crate (`hasPart`) or mainly referenced (`mentions`) and typed as _external_ to the Research Object (Contextual Entities).  It is worth pointing out that this semantic containment can extend beyond the physical containment of files residing within the RO-Crate Root directory on a given storage system, as the RO-Crate data entities may include any data resource globally identifiable using IRIs.

### Extensibility and RO-Crate profiles

The RO-Crate specification provides a core set of conventions to describe research outputs using types and properties applicable across scientific domains. However we have found that domain-specific use of RO-Crate will, implicitly or explicitly, form a specialized **profile** of RO-Crate; _a set of conventions, types and properties that one minimally can require and expect to be present in that subset of RO-Crates_. For instance, RO-Crates used for exchange of workflows will have to contain a data entity of type `ComputationalWorkflow`, or cultural heritage records should have a `contentLocation`. 

These profiles allow further reliable programmatic consumption and generation of RO-Crates, 
Following the RO-Crate mantra of guidance over strictness, profiles are mainly _duck-typing_ rather than strict semantic types, but may also have corresponding machine-readable schemas at multiple levels (file formats, JSON, RDF shapes, RDFS/OWL semantics).

The next version of the RO-Crate specification 1.2 will define a [formalization](https://www.researchobject.org/ro-crate/1.2-DRAFT/profiles) for publishing and declaring conformance to RO-Crate profiles, and optionally define a machine-readable profile as a _Profile Crate_ — another RO-Crate that describe the profile and in addition can list schemas for validation, compatible software, accepting repositories, serialization/packaging formats, extension vocabularies, custom JSON-LD contexts and examples. (See for example the [Workflow RO-Crate profile](https://about.workflowhub.eu/Workflow-RO-Crate/ro-crate-preview.html))

In addition, there are sometimes existing domain-specific metadata formats already exist, but they are either not RDF-based (and thus challenging to add terms for in JSON-LD) or are at a different granularity level that might become overwhelming if represented directly in the RO-Crate Metadata file (e.g. W3C PROV bundle detailing a workflow run [@doi:10.1093/gigascience/giz095]). RO-Crate allow such alternative metadata files to co-exist, and be described as data entities with references to the standards and vocabularies they conform to, enabling their programmatic consumption even where no filename or file extension conventions have emerged for those metadata formats.

Section 4 _Profiles of RO-Crate in use_ examines the observed specialization of RO-Crate use in several domains and their emerging profiles.

[^1]: IRIs[@doi:10.17487/rfc3987] are a generalisation of URIs (which include well-known http/https URLs), permitting international Unicode characters without `%`-encoding, commonly used on the browser address bar and in HTML5.
