%% TODO: Being onverted to LaTex math mode. See GitHub rendering for original Unicode symbols
% https://github.com/stain/ro-crate-paper/blob/master/content/80.formal-definition.md

% Text remains for now as Markdown blocks


\begin{markdown}
## Formalizing RO-Crate in First Order Logic

Below is an attempt to formalize the concept of RO-Crate as a set of relations using First Order Logic:

### Language

\end{markdown}

Definition of language $\mathcal{L}_{rocrate}$:

\begin{eqnarray*}
    \mathcal{L}_{rocrate}   & \equiv & { Property(p), Class(c), 
                            Literal(x), \mathbb{R}, \mathbb{S} } \\
    \mathbb{D}              & \equiv & \mathbb{IRI} \\
    \mathbb{IRI}            & \equiv & { \text{IRIs as defined in RFC3987} } \\
    \mathbb{R}              & \equiv & { \text{real or integer numbers} } \\
    \mathbb{S}              & \equiv & { \text{literal strings} }
\end{eqnarray*}

\begin{markdown}
The domain of discourse $\mathbb{D}$ is the set of $\mathbb{IRI}$ identifiers [@doi:10.17487/rfc3987] (notation `<http://example.com/>`), with additional descriptions using numbers $\mathbb{R}$ (notation `13.37`) and literal strings $\mathbb{S}$ (notation `"Hello"`). 

From this formalised language $\mathcal{L}_{rocrate}$ we can interpret an RO-Crate in any representation that can gather these descriptions, their properties, classes, and literal attributes.  
\end{markdown}

\begin{markdown}
### Minimal RO-Crate

Below we use $\mathcal{L}_{rocrate}$ to define a minimal RO-Crate:
\end{markdown}

\begin{eqnarray*}
ROCrate(R)                                 & \models & Root(R) \land Mentions(R, R) \land hasPart(R, d) \land \\
                                            & & Mentions(R, d) \land DataEntity(d) \land \\
                                            & & Mentions(R, c) \land ContextualEntity(c) \\
\forall r Root(r)                           & \Rightarrow & Dataset(r) \land name(r, n) \land description(r, d) \land \\
                                            & &             published(r, date) \land license(e, l) \\
\forall e \forall n \ name(e, n)            & \Rightarrow & Literal(n) \\
\forall e \forall s \ description(e, s)     & \Rightarrow & Literal(s) \\
\forall e \forall d \ datePublished(e, d)   & \Rightarrow & Literal(d) \\
\forall e \forall l \ license(e, l)         & \Rightarrow & ContextualEntity(l) \\
DataEntity(e)                               & \equiv &      File(e) \oplus Dataset(e) \\
Entity(e)                                   & \equiv &      DataEntity(e) \lor ContextualEntity(e) \\
\forall e \ Entity(e)                       & \Rightarrow & Class(e) \\
Mentions(R, s)                              & \models &     Relation(s, p, e) \oplus Attribute(s,  p, l) \\
Relation(s, p, o)                           & \models &     Entity(s) \land Property(p) \land  Entity(o) \\
Attribute(s, p, x)                          & \models &     Entity(s) \land Property(p) \land Literal(x) \\
Literal(x)                                  & \equiv &      x \in \mathbb{R} \oplus x \in \mathbb{S}
\end{eqnarray*}

\begin{markdown}
An $ROCrate(R)$ is defined as a self-described _Root Data Entity_, which describes and contains parts (_data entities_), which are further described in _contextual entities_.  These terms align with their use in the [RO-Crate 1.1 terminology](https://www.researchobject.org/ro-crate/1.1/terminology). 

The $Root(r)$ is a type of $Dataset(r)$, and must have the metadata to literal attributes to provide a $name$, $description$ and $datePublished$, as well as a contextual entity identifying its license. These predicates correspond to the RO-Crate 1.1 [requirements for the root data entity](https://www.researchobject.org/ro-crate/1.1/root-data-entity.html#direct-properties-of-the-root-data-entity).

The concept of an $Entity(e)$ is introduced as being either a $DataEntity(e)$, a $ContextualEntity(e)$, or [both](https://www.researchobject.org/ro-crate/1.1/contextual-entities.html#contextual-vs-data-entities); and must be typed with at least one $Class(e)$. 

For simplicity in this formalization (and to assist production rules below) $R$ is a constant representing a single RO-Crate, typically written to independent RO-Crate Metadata files. $R$ is used by $Mentions(R, e)$ to indicate that $e$ is an Entity described by the RO-Crate and therefore its metadata (a set of $Relation$ and $Attribute$ predicates) form part of the RO-Crate serialization. $Relation(s, p, o)$ and $Attribute(s, p, x)$ are defined as a _subject-predicate-object_ triple pattern from an $Entity(s)$ using a $Property(p)$ to either another $Entity(o)$ or a $Literal(x)$ value.
\end{markdown}

\begin{markdown}
### Example of formalized RO-Crate 

The below is an example RO-Crate represented using the above formalization, assuming a base URI of `<http://example.com/ro/123/>`:

    ROCrate(<http://example.com/ro/123/>)
    name(<http://example.com/ro/123/, 
        “Data files associated with the manuscript:Effects of …”)
    description(<http://example.com/ro/123/, 
        “Palliative care planning for nursing home residents …")
    datePublished(<http://example.com/ro/123/>, “2017")
    license(<http://example.com/ro/123/>, 
        <https://creativecommons.org/licenses/by-nc-sa/3.0/au/>
    ContextualEntity(<https://creativecommons.org/licenses/by-nc-sa/3.0/au/>)
    name(<https://creativecommons.org/licenses/by-nc-sa/3.0/au/, 
        “Attribution-NonCommercial-ShareAlike 3.0 Australia (CC BY-NC-SA 3.0 AU)”)

    hasPart(<http://example.com/ro/123/>, <http://example.com/ro/123/file.txt>)
    File(<http://example.com/ro/123/survey.csv>)
    name(<http://example.com/ro/123/survey.csv>, “Survey of care providers”)
    hasPart(<http://example.com/ro/123/>, <http://www.example.om/ro/123/folder/>)
    Dataset(<http://example.com/ro/123/interviews/>)
    name(<http://example.com/ro/123/interviews/>, 
        “Audio recordings of care provider interviews”)
\end{markdown}

\begin{markdown}
In reality many additional attributes from schema.org types like <http://schema.org/Dataset> and <http://schema.org/CreativeWork> would be used to further describe the RO-Crate and its entities, but as these are optional they do not form part of this formalization.
\end{markdown}

\begin{markdown}
### Mapping to RDF with schema.org

A formalized RO-Crate can be mapped to different serializations. Below follows a mapping to RDF using schema.org.

    Dataset(d) →  type(d, <http://schema.org/Dataset>)
    File(f) →  type(f, <http://schema.org/MediaObject>)
    Property(p) →  type(p, <http://www.w3.org/2000/01/rdf-schema#Property>)
    Class(c) →  type(c, <http://www.w3.org/2000/01/rdf-schema#Class>)
    CreativeWork(e) →  ContextualEntity(e) ∧ type(e, <http://schema.org/CreativeWork>)
    hasPart(e, t) →  Relation(e, <http://schema.org/hasPart>, t)
    type(e, t) →  Relation(e, <http://www.w3.org/1999/02/22-rdf-syntax-ns#type>, t) ∧ Class(t)
    name(e, n) →  Attribute(e, <http://schema.org/name>, n)
    description(e, d) →  Attribute(e, <http://schema.org/description>, d)
    datePublished(e, date) →  Attribute(e, <http://schema.org/datePublished>, date)
    license(e, l) →  Relation(e, <http://schema.org/license>, l) ∧ CreativeWork(l)

Note that in the JSON-LD serialization of RO-Crate the expression of $Class$ and $Property $is typically indirect, as the JSON-LD $@context$ maps to schema.org IRIs, which when resolved as Linked Data embeds their formal definition as RDFa. 
\end{markdown}

\begin{markdown}

### RO-Crate 1.1 Metadata File Descriptor

An important RO-Crate principle is that of being **self-describing**. Therefore the serialization of the RO-Crate into a file should also describe itself in a [Metadata File Descriptor](https://www.researchobject.org/ro-crate/1.1/root-data-entity.html#ro-crate-metadata-file-descriptor), indicating it is about (describing) the RO-Crate root data entity, and that it conformsTo a particular version of the RO-Crate specification:

    about(s,o) →  Relation(s, <http://schema.org/about>, o)
    conformsTo(s,o) →  Relation(s, <http://purl.org/dc/terms/conformsTo>, R)
    MetadataFileDescriptor(m) →  ( CreativeWork(m) ∧ about(m,R) ∧ RO-Crate(R) ∧ 
        conformsTo(m, <https://w3id.org/ro/crate/1.1>) )

Note that although the metadata file necessarily is an _information resource_ written to disk or served over the network (e.g. as JSON-LD), it is not considered to be a contained _part_ of the RO-Crate in the form of a _data entity_, rather it is described only as a _contextual entity_.

While in the conceptual model the _RO-Crate Metadata File_ can be seen as the top-level node that describes the _RO-Crate Root_, in the formal model (and the JSON-LD format) the metadata file descriptor is an additional contextual entity and not affecting the depth-limit of the RO-Crate.
\end{markdown}

\begin{markdown}
### Forward-chained Production Rules for JSON-LD

Combining the above predicates and schema.org mapping with rudimentary JSON templates, these forward-chaining production rules can output JSON-LD according to the RO-Crate 1.1 specification[^2]:

    Mentions(R, s) ∧ Relation(s, p, o) →  Mentions(R, o)
    i ∈ $\mathbb{Iri}$ → "i"
    r ∈ ℝ →  r
    s ∈ $\mathbb{S}$ → "s"
    ∀s∀p∀o Relation(s,p,o) →  { "@id": s,
                                p: { "@id": o }
                              }     
    ∀s∀p∀v Attribute(s,p,v) →  { "@id": s,
                                p: v 
                               }
    ∀r∀c  RO-Crate(r) →  { "@graph": [ Mentions(r, c)* ] }
    R ⊨  <./>
    MetadataFileDescriptor(<ro-crate-metadata.json>) 

This exposes the first order logic domain of discourse of IRIs, with rational numbers and strings as their corresponding JSON-LD representation. These production rules first grow the graph of $R$ by adding a transitive rule – anything described in $R$ which is related to $o$, means that $o$ is also mentioned by the $ROCrate(R)$. For simplicity this rule is one-way; in practice the JSON-LD graph can also contain free-standing contextual entities that have outgoing relations to data- and contextual entities.

[^2]:
    Limitations: The full list of types, relations and attribute properties from the RO-Crate specification are not included. Examples shown include $datePublished$, $CreativeWork$ and $name$. Contextual entities not related from the RO-Crate (e.g. using inverse relations to a data entity) would not be covered by the single direction $Mentions(R, s)$ production rule; see [issue 122](https://github.com/ResearchObject/ro-crate/issues/122). The $datePublished(e, date)$ rule do not include syntax checks for the ISO 8601 datetime format. Compared with RO-Crate examples, this generated JSON-LD does not use a $@context$ as the IRIs are produced unshortened; a post-step could do JSON-LD Flattening with a versioned RO-Crate context.
\end{markdown}
