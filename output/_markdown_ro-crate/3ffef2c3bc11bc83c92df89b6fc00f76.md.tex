\markdownRendererHeadingTwo{Regulatory Sciences}\markdownRendererInterblockSeparator
{}\markdownRendererLink{BioCompute Objects}{https://www.biocomputeobject.org/}{https://www.biocomputeobject.org/}{} (BCO) \markdownRendererCite{1}+{}{}{doi:10.1371/journal.pbio.3000099} is a community-led effort to standardise submissions of computational workflows to biomedical regulators. For instance, a genomics sequencing pipeline, as part of a personalised cancer treatment study, can be submitted to the US Food and Drugs Administration (FDA) for approval. BCOs are formalised in the standard IEEE 2791-2020 \markdownRendererCite{1}+{}{}{doi:10.1109/IEEESTD.2020.9094416} as a combination of \markdownRendererLink{JSON Schemas}{https://opensource.ieee.org/2791-object/ieee-2791-schema/}{https://opensource.ieee.org/2791-object/ieee-2791-schema/}{} that define the structure of JSON metadata files describing exemplar workflow runs in detail, covering aspects such as the usability and error domain of the workflow, its runtime requirements, the reference datasets used and representative output data produced.\markdownRendererInterblockSeparator
{}BCOs provide a structured view over a particular workflow, informing regulators about its workings independently of the underlying workflow definition language. However, BCOs have only limited support for additional metadata\markdownRendererFootnote{IEEE 2791-2020 do permit user extensions in the \markdownRendererEmphasis{extension domain} by referencing additional JSON Schemas.}. For instance, while the BCO itself can indicate authors and contributors, and in particular regulators and their review decisions, it cannot describe the provenance of individual data files or workflow definitions. \markdownRendererInterblockSeparator
{}As a custom JSON format, BCOs cannot be extended with Linked Data concepts, except by adding an additional top-level JSON object formalised in another JSON Schema. A BCO and workflow submitted by upload to a regulator will also frequently consist of multiple cross-related files. Crucially, there is no way to tell whether a given \markdownRendererCodeSpan{*.json} file is a BCO file, except by reading its content and check for its \markdownRendererCodeSpan{spec\markdownRendererUnderscore{}version}. \markdownRendererInterblockSeparator
{}We can then consider how a BCO and its referenced artefacts can be packaged and transferred following FAIR principles. \markdownRendererLink{\markdownRendererStrongEmphasis{BCO RO-Crate}}{https://biocompute-objects.github.io/bco-ro-crate/}{https://biocompute-objects.github.io/bco-ro-crate/}{}\markdownRendererCite{1}+{}{}{doi:10.5281/zenodo.4633732}, part of the BioCompute Object user guides, defines a set of best practices for wrapping a BCO with a workflow, together with its exemplar outputs in an RO-Crate, which then provides typing and additional provenance metadata of the individual files, workflow definition, referenced data and the BCO metadata itself. \markdownRendererInterblockSeparator
{}Here the BCO is responsible for describing the \markdownRendererEmphasis{purpose} of a workflow and its run at an abstraction level suitable for a domain scientist, while the more open-ended RO-Crate describes the surroundings of the workflow, classifying and relating its resources and providing provenance of their existence beyond the BCO. This emerging \markdownRendererEmphasis{separation of concerns} highlight how RO-Crate is used side-by-side of existing standards, even where there are apparent partial overlaps.\markdownRendererInterblockSeparator
{}A similar separation of concerns can be found if considering the RO-Crate as a set of files, where the \markdownRendererEmphasis{transport-level} metadata, such as checksum of files, are \markdownRendererLink{delegated to BagIt}{https://www.researchobject.org/ro-crate/1.1/appendix/implementation-notes.html#adding-ro-crate-to-bagit}{https://www.researchobject.org/ro-crate/1.1/appendix/implementation-notes.html#adding-ro-crate-to-bagit}{} manifests, a standard focusing on the preservation challenges of digital libraries\markdownRendererCite{1}+{}{}{doi:10.17487/rfc8493}. As such, RO-Crates are not required to iterate all the files in their folder hierarchy, only those that benefit from being described.\markdownRendererInterblockSeparator
{}Specifically, a BCO alone is insufficient for reliable re-execution of a workflow, which would need a compatible workflow engine depending on the workflow definition language, so IEEE 2791 recommends using Common Workflow Language \markdownRendererCite{1}+{}{}{arxiv:2105.07028} for interoperable pipeline execution. CWL itself relies on tool packaging in software containers using \markdownRendererLink{Docker}{https://www.docker.com/}{https://www.docker.com/}{} or \markdownRendererLink{Conda}{https://docs.conda.io/}{https://docs.conda.io/}{}. Thus, we can consider BCO RO-Crate as a stack: transport-level manifests of files (BagIt), provenance, typing and context of those files (RO-Crate), workflow overview and purpose (BCO), interoperable workflow definition (CWL) and tool distribution (Docker).\markdownRendererInterblockSeparator
{}\markdownRendererImage{separationofconcerns}{../content/images/ro-crate-bco-sep-of-concerns.pdf}{../content/images/ro-crate-bco-sep-of-concerns.pdf}{\textbf{Separation of Concerns in BCO RO-Crate}. BioCompute Object (IEEE2791) is a JSON file that structurally explains the purpose and implementation of a computational workflow, for instance implemented in Nextflow, that installs the workflow’s software dependencies as Docker containers or BioConda packages. An example execution of the workflow shows the different kinds of result outputs, which may be external, using GitHub LFS to support larger data. RO-Crate gathers all these local and external resources, relating them and giving individual descriptions, for instance permanent DOI identifiers for reused datasets accessed from Zenodo, but also adding external identifiers to attribute authors using ORCID or to identify which licences apply to individual resources. The RO-Crate and its local files are captured in a BagIt whose checksum ensures completeness, combined with Big Data Bag \cite{doi:10.1109/BigData.2016.7840618} features to “complete” the bag with large external files such as the workflow outputs}\relax