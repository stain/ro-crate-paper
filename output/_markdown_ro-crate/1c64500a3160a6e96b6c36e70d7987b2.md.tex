\markdownRendererHeadingTwo{Conceptual Definition}\markdownRendererInterblockSeparator
{}A key premise of RO-Crate is the existence of a wide variety of resources on the Web that can help describe research. As such, RO-Crate relies on concepts from the Web, in particular that of Linked Data \markdownRendererCite{1}+{}{}{doi:10.2200/S00334ED1V01Y201102WBE001}. Figure \ref{fig:conceptual} shows the main conceptual elements involved in an RO-Crate; an RO-Metadata File(top) describes the research object using structured metadata including external references, coupled with the contained artefacts (bottom) bundled and described by the RO-Crate.\markdownRendererInterblockSeparator
{}\markdownRendererImage{Conceptual RO-Crate Overview}{../content/images/ro-crate-overview.pdf}{../content/images/ro-crate-overview.pdf}{\textbf{Conceptual overview of RO-Crate}. A \emph{Persistent Identifier} (PID) [@doi:10.1371/journal.pbio.2001414] identifies a \emph{Research Object} (RO), which is archived using BagIt \cite{doi:10.17487/rfc8493}, OCFL \cite{ocfl_2020}, git or ZIP. The RO is described within a \textit{RO Metadata File}, providing identifiers for authors using ORCID, organisations using [ROR](https://ror.org/) and licences such as Creative Commons. The \emph{RO-Crate content} is further described with its own metadata. Data can be embedded files and directories, as well as links to external web resources, PIDs [@10.1371/journal.pbio.2001414] and nested RO-Crates.}{#fig:conceptual width="90%"}\markdownRendererInterblockSeparator
{}\markdownRendererHeadingThree{Linked Data as a core principle}\markdownRendererInterblockSeparator
{}Linked Data principles \markdownRendererCite{1}+{}{}{doi:10.4018/978-1-60960-593-3.ch008} (use of IRIs to identify resources (i.e. artefacts), resolvable via HTTP, enriched with metadata and linked to each other) are core to RO-Crate; therefore IRIs\markdownRendererFootnote{IRIs\markdownRendererCite{1}+{}{}{doi:10.17487/rfc3987} are a generalisation of URIs (which include well-known http/https URLs), permitting international Unicode characters without \markdownRendererCodeSpan{\markdownRendererPercentSign{}}-encoding, commonly used on the browser address bar and in HTML5.} are used to identify an RO-Crate, its constituent parts and metadata descriptions, and the properties and classes used in the metadata. \markdownRendererInterblockSeparator
{}Linked Data make it possible for consumers to follow links for more (ideally both human- and machine-readable) information; as the RO-Crate relies on these principles, it can be sufficiently \markdownRendererEmphasis{self-describing} as artefacts can be interrelated using global identifiers, without needing to recursively fully describe every referenced artefact.\markdownRendererInterblockSeparator
{}[PROPOSED REPHRASE ON PREVIOUS PAR] RO Crates are \markdownRendererEmphasis{self-described}; and follow the Linked Data principles to describe all of their resources in both human and machine readable manner. Hence, resources are identified using global identifiers; and relationships between two resources are defined with links.\markdownRendererInterblockSeparator
{}The foundation on Linked Data and shared vocabularies also means multiple RO-Crates and other Linked Data resources can be indexed, combined, queried, integrated or transformed using existing Semantic Web technologies such as \markdownRendererLink{SPARQL}{https://www.w3.org/TR/sparql11-overview/}{https://www.w3.org/TR/sparql11-overview/}{} and knowledge graph triple stores.\markdownRendererInterblockSeparator
{}\markdownRendererHeadingThree{RO-Crate is a self-described container}\markdownRendererInterblockSeparator
{}An RO-Crate is defined as a self-described \markdownRendererStrongEmphasis{Root Data Entity} that describes and contains \markdownRendererEmphasis{data entities}, which are further described using \markdownRendererEmphasis{contextual entities}. A \markdownRendererStrongEmphasis{data entity} is either a \markdownRendererEmphasis{file} (i.e. a set of bytes stored on disk somewhere) or a \markdownRendererEmphasis{directory} (i.e. dataset of named files and other directories) — while a \markdownRendererStrongEmphasis{contextual entity} exists outside the information system (e.g. a Person, a workflow language) and it is defined by its metadata. The representation of a \markdownRendererStrongEmphasis{data entity} as a set of bytes makes it possible to store a variety of research artefacts including not only data but also, for instance, software and text.\markdownRendererInterblockSeparator
{}The Root Data Entity is a directory, the RO-Crate root, identified by the presence of the \markdownRendererStrongEmphasis{RO-Crate Metadata File} (\markdownRendererCodeSpan{ro-crate-metadata.json}). This is a JSON-LD file that describes the RO-Crate, its content and related metadata using Linked Data. JSON-LD is an RDF serialisation that has become popular as it is easy to read by humans while also offers some advantages for data exchange on the Internet (e.g. it removes some of the cross-domain restrictions that can be present with XML). JSON-LD is the preferred and widely supported format by RO-Crate tools and community.\markdownRendererInterblockSeparator
{}The minimal \markdownRendererLink{requirements for the root data entity metadata}{https://www.researchobject.org/ro-crate/1.1/root-data-entity.html#direct-properties-of-the-root-data-entity}{https://www.researchobject.org/ro-crate/1.1/root-data-entity.html#direct-properties-of-the-root-data-entity}{} are \markdownRendererCodeSpan{name}, \markdownRendererCodeSpan{description} and \markdownRendererCodeSpan{datePublished}, as well as a contextual entity identifying its \markdownRendererCodeSpan{license} — additional metadata are frequently added depending on the purpose of the particular RO-Crate.\markdownRendererInterblockSeparator
{}RO-Crate can be stored, transferred or published in multiple ways, such as BagIt \markdownRendererCite{1}+{}{}{doi:10.17487/rfc8493}, Oxford Common File Layout \markdownRendererCite{1}+{}{}{ocfl_2020} (OCFL), downloadable ZIP archives in Zenodo or through dedicated online repositories, as well as published directly on the Web, e.g. using GitHub Pages. Combined with Linked Data identifiers, this caters for a diverse set of storage and access requirements across different scientific domains, from metagenomics workflows producing hundreds of gigabytes of genome data to cultural heritage records with access restrictions for personally identifiable data.\markdownRendererInterblockSeparator
{}\markdownRendererHeadingThree{Data Entities are described using Contextual Entities}\markdownRendererInterblockSeparator
{}RO-Crate distinguishes between \markdownRendererLink{data and contextual entities}{https://www.researchobject.org/ro-crate/1.1/contextual-entities.html#contextual-vs-data-entities}{https://www.researchobject.org/ro-crate/1.1/contextual-entities.html#contextual-vs-data-entities}{} in a similar way to HTTP terminology's early attempt to separate \markdownRendererEmphasis{information} (data) and \markdownRendererEmphasis{non-information} (contextual) resources \markdownRendererCite{1}+{}{}{httprange14}. Data entities are usually files and directories located by relative IRI references within the RO-Crate Root, but they can also be Web or restricted resources identified with absolute IRIs.\markdownRendererInterblockSeparator
{}As both types of entities are identified by IRIs, their distinction is allowed to be blurry; data entities can be located anywhere and be complex, while contextual entities can have a Web presence beyond their description inside the RO-Crate. For instance \markdownRendererCodeSpan{https://orcid.org/0000-0002-1825-0097} is primarily an identifier for a person, but secondarily also a web page that describes that person and their academic work. \markdownRendererInterblockSeparator
{}Any particular IRI might appear as a contextual entity in one RO-Crate and as a data entity in another; their distinction lies in the fact that data entities can be considered to be \markdownRendererEmphasis{contained} or captured by the RO-Crate, while contextual entities mainly \markdownRendererEmphasis{explain} the RO-Crate and its entities. It follows that an RO-Crate should have one or more data entities, although this is not a formal requirement. \markdownRendererInterblockSeparator
{}Figure \ref{fig:uml} shows a UML view of RO-Crate, highlighting the different types of data entities and contextual entities that can be aggregated and related.\markdownRendererInterblockSeparator
{}\markdownRendererImage{RO-Crate UML}{../content/images/ro-crate-uml.pdf}{../content/images/ro-crate-uml.pdf}{\textbf{UML model view of RO-Crate.} The \emph{RO-Crate Metadata File} conforms to a version of the specification, which mainly describes the \emph{RO-Crate Root Data Entity} representing the Research Object as a dataset. The RO-Crate aggregates \emph{data entities} (\texttt{hasPart}) which are further described using \emph{contextual entities}. Multiple types and relations from Schema.org allow annotations to be more specific, including figures, nested datasets, computational workflows, people, organisations, instruments and places.}{#fig:uml width="90%"}\markdownRendererInterblockSeparator
{}\markdownRendererHeadingThree{Best Practice Guide rather than strict specification}\markdownRendererInterblockSeparator
{}RO-Crate builds on Linked Data standards (JSON-LD \markdownRendererCite{1}+{}{}{sporny_2014}) and community-evolved vocabularies (\markdownRendererLink{Schema.org}{https://schema.org/}{https://schema.org/}{} \markdownRendererCite{1}+{}{}{doi:10.1145/2857274.2857276}). RO-Crate aims to build a set of best practices on how to practically apply these existing standards in a common way to describe research outputs and their provenance, without having to learn each of the underlying technologies in detail.\markdownRendererInterblockSeparator
{}As such, the \markdownRendererLink{RO-Crate 1.1}{https://w3id.org/ro/crate/1.1}{https://w3id.org/ro/crate/1.1}{} specification \markdownRendererCite{1}+{}{}{doi:10.5281/zenodo.4541002} can be seen as an opinionated and example-driven guide to writing Schema.org metadata as JSON-LD, which leaves it open for implementers to include additional metadata using other Schema.org types and properties, or even additional Linked Data vocabularies/ontologies or their own ad-hoc terms.\markdownRendererInterblockSeparator
{}\markdownRendererHeadingThree{Checking Simplicity}\markdownRendererInterblockSeparator
{}As previously stated, one aim of RO-Crate is to be conceptually simple. This simplicity has been repeatedly checked and confirmed through a community review process. For instance, in the discussion on supporting \markdownRendererLink{ad-hoc vocabularies}{https://github.com/ResearchObject/ro-crate/issues/71}{https://github.com/ResearchObject/ro-crate/issues/71}{} in RO-Crate, the community explored potential Linked Data solutions. The conventional wisdom in \markdownRendererLink{RDF best practice}{https://www.w3.org/TR/swbp-vocab-pub/}{https://www.w3.org/TR/swbp-vocab-pub/}{} is to establish a vocabulary with a new URI namespace, formalised using \markdownRendererLink{RDF Schema}{http://www.w3.org/TR/2014/REC-rdf-schema-20140225/}{http://www.w3.org/TR/2014/REC-rdf-schema-20140225/}{} or \markdownRendererLink{OWL}{http://www.w3.org/TR/2012/REC-owl2-overview-20121211/}{http://www.w3.org/TR/2012/REC-owl2-overview-20121211/}{} ontologies, however, as this may seem excessive learning curve for the use case of clarifying that \markdownRendererCodeSpan{sentence} in an RO-Crate refers to prisoner sentencing rather than a linguistic structure, the RO-Crate community instead agreed on a dual lightweight approach: (ⅰ) \markdownRendererLink{Document}{https://www.researchobject.org/ro-crate/1.1/appendix/jsonld.html#adding-new-or-ad-hoc-vocabulary-terms}{https://www.researchobject.org/ro-crate/1.1/appendix/jsonld.html#adding-new-or-ad-hoc-vocabulary-terms}{} how projects with their own web-presence can make a pure HTML-based vocabulary, and (ⅱ) provide a community-wide PID namespace under \markdownRendererLink{https://w3id.org/ro/terms/}{https://w3id.org/ro/terms/}{https://w3id.org/ro/terms/}{} that redirect to simple CSV files \markdownRendererLink{maintained in GitHub}{https://github.com/ResearchObject/ro-terms}{https://github.com/ResearchObject/ro-terms}{}. \markdownRendererInterblockSeparator
{}To further verify this idea, we have formalised the RO-Crate definition (see Appendix A). An important result of this exercise is that the underlying data structure of RO-Crate, although conceptually a graph, is represented as a tree limited to depth 2. The formalisation also emphasises the \markdownRendererEmphasis{boundedness} of the structure; namely, the fact that elements are specifically identified as being either semantically \markdownRendererEmphasis{contained} by the RO-Crate (\markdownRendererCodeSpan{hasPart}) or mainly referenced (\markdownRendererCodeSpan{mentions}) and typed as \markdownRendererEmphasis{external} to the Research Object (Contextual Entities). \relax