\markdownRendererHeadingOne{Profiles of RO-Crate in use}\markdownRendererInterblockSeparator
{}RO-Crate is fundamentally an infrastructure to help build FAIR research artefacts. In other words, the key question is whether RO-Crate can be used to share and (re)use research artefacts. Here we look at three research domains where RO-Crate is being applied: Bioinformatics, Regulatory Science and Cultural Heritages. In addition, we note how RO-Crate may have an important role as part of machine-actionable data management plans and institutional repositories.\markdownRendererInterblockSeparator
{}From these varied uses of RO-Crate we observe a natural differences in their detail level and the type of entities described by the RO-Crate. For instance, on submission of an RO-Crate to a workflow repository, it is reasonable to expect the RO-Crate to contain at least one workflow, ideally with a declared licence and workflow language. Specific additional recommendations such as on identifiers is also needed to meet the emerging requirements of \markdownRendererLink{FAIR Digital Objects}{https://fairdo.org/}{https://fairdo.org/}{}. \markdownRendererLink{Work has now begun}{https://github.com/ResearchObject/ro-crate/issues/153}{https://github.com/ResearchObject/ro-crate/issues/153}{} to formalise these different \markdownRendererEmphasis{profiles} of RO-Crates, which may impose additional constraints based on the needs of a specific domain or use case. \relax