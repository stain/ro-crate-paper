\markdownRendererHeadingOne{RO-Crate Tooling}\markdownRendererInterblockSeparator
{}The work of the community led to the development of a number of tools for creating and using RO-Crates. Table \ref{tab:tools} shows the current set of implementations. Reviewing this list, one can see that tools support commonly used programming languages, including Python, JavaScript, and Ruby. Additionally, these tools can be integrated into commonly used research environments; in particular, the command line (\markdownRendererEmphasis{ro-crate-html-js}). Furthermore, there are tools that cater to the end-user (\markdownRendererEmphasis{Describo}, \markdownRendererEmphasis{Workflow Hub}). For example, Describo was developed to help researchers of the Australian \markdownRendererLink{Criminal Characters project}{https://criminalcharacters.com/}{https://criminalcharacters.com/}{} annotate historical prisoner records to gain greater insight into the history of Australia \markdownRendererCite{1}+{}{}{doi:10.1080/14490854.2020.1796500}. \markdownRendererInterblockSeparator
{}While the development of these tools is promising, our analysis of their maturity status shows that the majority of them are in the Beta stage. This is partly due to the fact that the RO-Crate specification itself only recently reached 1.0 status, in November 2019 \markdownRendererCite{1}+{}{}{doi:10.5281/zenodo.3541888}. Now that there is a fixed point of reference, and RO-Crate 1.1 (October 2020) \markdownRendererCite{1}+{}{}{doi:10.5281/zenodo.4031327} has stabilised based on feedback from application development, we expect to see a further increase in the maturity of these tools, along with the creation of new ones.\markdownRendererInterblockSeparator
{}Given the stage of the specification, these tools have been primarily targeted to developers, essentially providing them with the core libraries for working with RO-Crates. Another target has been that of research data managers who need to manage and curate large amounts of data. \relax