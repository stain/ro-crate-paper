\markdownRendererHeadingTwo{Technical implementation of the model}\markdownRendererInterblockSeparator
{}The RO-Crate conceptual model has been realised using JSON-LD and Schema.org in a prescriptive form as discussed in the best practice approach. This technical approach again caters for simplicity. \markdownRendererInterblockSeparator
{}\markdownRendererLink{JSON-LD}{https://json-ld.org/}{https://json-ld.org/}{} provides a way to express Linked Data as a JSON structure, where a \markdownRendererEmphasis{context} provides mapping to RDF properties and classes. While JSON-LD cannot map arbitrary JSON structures to RDF, we found it does lower the barrier regarding Linked Data serialisations as it follows a JSON structure, nowadays a very common and popular format for data exchange.\markdownRendererInterblockSeparator
{}However, JSON-LD alone, has too many degrees of freedom and hidden complexities for software developers to reliably produce and consume without specialised expertise or software libraries. A large part of the RO-Crate specification is therefore dedicated to describing JSON structures. \markdownRendererInterblockSeparator
{}\markdownRendererHeadingThree{RO-Crate JSON-LD}\markdownRendererInterblockSeparator
{}RO-Crate mandates the use of \markdownRendererLink{flattened, compacted JSON-LD}{https://www.researchobject.org/ro-crate/1.1/appendix/jsonld.html}{https://www.researchobject.org/ro-crate/1.1/appendix/jsonld.html}{} where a single \markdownRendererCodeSpan{@graph} array contains all the data and contextual entities in a flat list. An example can be seen in the JSON-LD snippet below, describing a simple RO-Crate containing two datasets (data1.txt and data2.txt):\markdownRendererInterblockSeparator
{}\markdownRendererInputFencedCode{../output/_markdown_ro-crate/634f7904aa7296f7ec26373a016d51ed.verbatim}{json}\markdownRendererInterblockSeparator
{}\markdownRendererEmphasis{Figure X}: \markdownRendererEmphasis{Simplified RO-Crate JSON-LD showing the flattened compacted \markdownRendererCodeSpan{@graph} array}\markdownRendererInterblockSeparator
{}It can be argued that this is a more graph-like approach than the tree structure JSON would otherwise invite, and which is normally emphasised as a feature of JSON-LD in order to “hide” its RDF nature. \markdownRendererInterblockSeparator
{}However, we found that the use of trees for, e.g. \markdownRendererEmphasis{a \markdownRendererEmphasis{Person} entity appearing as author of a \markdownRendererCodeSpan{File} which nests under a \markdownRendererCodeSpan{Dataset}, \markdownRendererCodeSpan{hasPart}}, counter-intuitively leads one to consider the JSON-LD as an RDF Graph, since an identified \markdownRendererCodeSpan{Person} entity can then appear at multiple and repeated points of the tree (e.g. author of multiple files), necessitating node merging or duplication. \markdownRendererInterblockSeparator
{}By comparison, a single flat \markdownRendererCodeSpan{@graph} array approach means that applications can process and edit each entity as pure JSON by a simple lookup based on \markdownRendererCodeSpan{@id}. At the same time, lifting all entities to the same level emphasises Research Object's principle that describing the context and provenance is just as important as describing the data.\markdownRendererInterblockSeparator
{}RO-Crate reuses Schema.org, but provides its own versioned JSON-LD context, which has a similar flat list with the mapping from JSON-LD keys to their URI equivalents (e.g. \markdownRendererCodeSpan{author} maps to \markdownRendererLink{http://schema.org/author}{http://schema.org/author}{http://schema.org/author}{}). The rationale behind this decision is to support JSON-based RO-Crate applications that are largely unaware of JSON-LD, and thus still may want to process the \markdownRendererCodeSpan{@context} to find Linked Data definitions of unknown properties and types. Not reusing the official Schema.org context means RO-Crate is also able to map in additional vocabularies where needed, namely the \markdownRendererEmphasis{Portland Common Data Model} (PCDM) \markdownRendererCite{1}+{}{}{pcdm} for repositories and Bioschemas \markdownRendererCite{1}+{}{}{bioschemas_2017} for describing computational workflows.\markdownRendererInterblockSeparator
{}Similarly, rather than relying on implications from \markdownRendererCodeSpan{"@type": "@id"} annotations in the context, RO-Crate JSON-LD distinguishes explicitly between references to other entities (\markdownRendererCodeSpan{\markdownRendererLeftBrace{}"@id": "\markdownRendererHash{}alice"\markdownRendererRightBrace{}}) and string values (\markdownRendererCodeSpan{"Alice"}) - meaning RO-Crate applications can find the corresponding entity without parsing the \markdownRendererCodeSpan{@context}.\relax