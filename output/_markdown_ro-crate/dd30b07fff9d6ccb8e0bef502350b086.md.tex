\markdownRendererHeadingTwo{Machine-actionable Data Management Plans}\markdownRendererInterblockSeparator
{}Machine-actionable Data Management Plans (maDMPs) have been proposed as an improvement to automate FAIR data management tasks in research \markdownRendererCite{1}+{}{}{doi:10.1371/journal.pcbi.1006750}, e.g. by using PIDs and controlled vocabularies to describe what happens to data over the research life cycle \markdownRendererCite{1}+{}{}{doi:10.1007/978-3-030-45442-5_15}. The Research Data Alliance's \markdownRendererEmphasis{DMP Common Standard} for maDMPs \markdownRendererCite{1}+{}{}{doi:10.15497/rda00039} is one such formalisation for expressing maDMPs, which can be expressed as Linked Data using the DMP Common Standard Ontology \markdownRendererCite{1}+{}{}{doi:10.4126/frl01-006423289}, a specialisation of the W3C Data Catalog Vocabulary (DCAT) \markdownRendererCite{1}+{}{}{dcat2}. RDA maDMPs are usually expressed using regular JSON, conforming to the DMP JSON Schema.\markdownRendererInterblockSeparator
{}A mapping has been produced between Research Object Crates and Machine-actionable Data Management Plans \markdownRendererCite{1}+{}{}{doi:10.4126/frl01-006423291}, implemented by the RO-Crate {RDA maDMP Mapper \markdownRendererCite{1}+{}{}{doi:10.5281/zenodo.3922136}. A similar mapping has been implemented by \markdownRendererCodeSpan{RO-Crate\markdownRendererUnderscore{}2\markdownRendererUnderscore{}ma-DMP} \markdownRendererCite{1}+{}{}{doi:10.5281/zenodo.3903463}. In both cases, a maDMP can be converted to a RO-Crate, or vice versa. In \markdownRendererCite{1}+{}{}{doi:10.4126/frl01-006423291} this functionality caters for two use cases:\markdownRendererInterblockSeparator
{}\markdownRendererOlBeginTight
\markdownRendererOlItemWithNumber{1}Start a skeleton data management plan based on an existing RO-Crate dataset, e.g. from an RO-Crate from WorkflowHub.\markdownRendererOlItemEnd 
\markdownRendererOlItemWithNumber{2}Instantiate an RO-Crate based on a data management plan.\markdownRendererOlItemEnd 
\markdownRendererOlEndTight \markdownRendererInterblockSeparator
{}An important difference here is that data management plans are (ideally) written in advance of data production, while RO-Crates are typically created to describe data after it has been generated. This approach shows the importance of \markdownRendererEmphasis{templating} to make both tasks more automatable and achievable, and how RO-Crate can fit into earlier stages of the research life cycle.\relax