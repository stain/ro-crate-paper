\markdownRendererHeadingTwo{Institutional data repositories – Harvard Data Commons}\markdownRendererInterblockSeparator
{}The concept of a Data Commons for research collaboration was originally defined as "cyber-infrastructure that co-locates data, storage, and computing infrastructure with commonly used tools for analysing and sharing data to create an interoperable resource for the research community" \markdownRendererCite{1}+{}{}{doi:10.1109/MCSE.2016.92}. More recently, it was established to integrate active data-intensive research with data management and archival best practices. It facilitates research by providing computational infrastructure where researchers can use, share and store data, software, workflows and other digital artefacts used in their studies. Furthermore, the Commons feature tools and services, such as computation clusters and storage for scalability, data repositories for disseminating and preserving regular, but also large or sensitive datasets, and other research assets. Multiple initiatives were undertaken to create Data Commons on national, research, and institutional levels. For example, the \markdownRendererLink{Australian Research Data Commons (ARDC)}{https://ardc.edu.au}{https://ardc.edu.au}{} \markdownRendererCite{1}+{}{}{doi:10.5334/dsj-2019-044} is a national initiative that enables local researchers and industries to access computing infrastructure, training, and curated datasets for data-intensive research. NCI’s \markdownRendererLink{Genomic Data Commons}{https://gdc.cancer.gov/}{https://gdc.cancer.gov/}{} (GDC) \markdownRendererCite{1}+{}{}{doi:10.1182/blood-2017-03-735654} provides the cancer research community with access to a vast volume of genomic and clinical data. Initiatives such as \markdownRendererLink{Research Data Alliance (RDA) Global Open Research Commons}{https://www.rd-alliance.org/groups/global-open-research-commons-ig}{https://www.rd-alliance.org/groups/global-open-research-commons-ig}{} propose standards on the implementation of Data Commons to avoid them becoming "data silos" and enable interoperability from one Data Commons to another.\markdownRendererInterblockSeparator
{}\markdownRendererStrongEmphasis{Harvard Data Commons} \markdownRendererCite{1}+{}{}{doi:10.7557/5.5422} aims to address data access and reuse challenges of cross-disciplinary research within a research institution. It brings together multiple institutional schools, libraries, computing centres and the \markdownRendererLink{Harvard Dataverse data repository}{https://dataverse.harvard.edu/}{https://dataverse.harvard.edu/}{}. \markdownRendererLink{Dataverse}{https://dataverse.org/}{https://dataverse.org/}{} \markdownRendererCite{1}+{}{}{doi:10.1045/january2011-crosas} is a free and open-source software platform to archive, share and cite research data. The Harvard Dataverse repository is the largest of 70 installations worldwide, containing over 100K datasets with about 1M data files. Toward the goal of facilitating collaboration and data discoverability and management within the university, Harvard Data Commons has the following primary objectives:\markdownRendererInterblockSeparator
{}\markdownRendererOlBeginTight
\markdownRendererOlItemWithNumber{1}integrating Harvard Research Computing with Harvard Dataverse by leveraging Globus endpoints \markdownRendererCite{1}+{}{}{doi:10.1109/MCC.2014.52} that will allow an automatic transfer of large datasets to the repository. In some cases, only the metadata will be transferred while the data stays stored in remote storage;\markdownRendererOlItemEnd 
\markdownRendererOlItemWithNumber{2}supporting advanced research workflows and providing packaging options for assets such as code and workflows in the Harvard Dataverse repository to enable reproducibility and reuse, and \markdownRendererOlItemEnd 
\markdownRendererOlItemWithNumber{3}integrating repositories supported by Harvard, which are \markdownRendererLink{DASH}{https://dash.harvard.edu}{https://dash.harvard.edu}{}, the open access institutional repository, the Digital Repository Services (DRS) for preserving digital asset collections, and the Harvard Dataverse.\markdownRendererOlItemEnd 
\markdownRendererOlEndTight \markdownRendererInterblockSeparator
{}Particularly relevant to this paper is the second objective of the Harvard Data Commons, which aims to support the deposit of research artefacts to Harvard Dataverse with sufficient information in the metadata to allow their future reuse (Figure~\ref{fig:hdc}). Considering the requirements of incorporating data, code, and other artefacts from various institutional infrastructures, Harvard Data Commons is currently working on RO-Crate adaptation. The RO-Crate metadata provides the necessary structure to make all research artefacts FAIR. The Dataverse software already has extensive support for metadata, including the Data Documentation Initiative (DDI), Dublin Core, DataCite, and Schema.org. Incorporating RO-Crate, which has the flexibility to describe a wide range of research resources, will facilitate their seamless transition from one infrastructure to the other within the Harvard Data Commons.\markdownRendererInterblockSeparator
{}\markdownRendererImage{Harvard Data Commons}{../content/images/data-commons-ro-crate-figure-2.pdf}{../content/images/data-commons-ro-crate-figure-2.pdf}{\textbf{One aspect of Harvard Data Commons}. Automatic encapsulation and deposit of artefacts from data management tools used during active research at the Harvard Dataverse repository.}{#fig:hdc}\markdownRendererInterblockSeparator
{}Even though the Harvard Data Commons is specific to Harvard University, the overall vision and the three objectives can be abstracted and applied to other universities or research organisations. The Commons will be designed and implemented using standards and commonly-used approaches to make it interoperable and reusable by others.\relax