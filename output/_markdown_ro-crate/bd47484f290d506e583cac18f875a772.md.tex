\markdownRendererHeadingOne{Introduction}\markdownRendererInterblockSeparator
{}The move towards open science has increased the need and demand for the publication of artefacts of the research process \markdownRendererCite{1}+{}{}{sefton_blog_post_2021}. This is particularly apparent in domains that rely on computational experiments; for example, the publication of software, datasets and records of the dependencies that such experiments rely on \markdownRendererCite{1}+{}{}{doi:10.1126/science.aah6168}. \markdownRendererInterblockSeparator
{}It is often argued that the publication of these assets, and specifically software \markdownRendererCite{1}+{}{}{doi:10.3233/DS-190026}, workflows \markdownRendererCite{1}+{}{}{https://doi.org/10.1162/dint_a_00033} and data, should follow the FAIR principles \markdownRendererCite{1}+{}{}{doi:10.1038/sdata.2016.18}; namely, that they are Findable, Accessible, Interoperable and Reusable. These principles are agnostic to the \markdownRendererEmphasis{implementation} strategy needed to comply with them. Hence, there has been an increasing amount of work in the development of platforms and specifications that aim to fulfil these goals \markdownRendererCite{1}+{}{}{isbn:9781315351148}. Important examples include data publication with rich metadata (e.g. Zenodo \markdownRendererCite{1}+{}{}{doi:10.3897/biss.3.37080}), domain-specific data deposition (e.g., PDB \markdownRendererCite{1}+{}{}{doi:10.1093/nar/gkl971}) and following practices for reproducible research software \markdownRendererCite{1}+{}{}{doi:10.1371/journal.pcbi.1003285} (e.g. use of containers). \markdownRendererInterblockSeparator
{}These strategies are focused primarily on one \markdownRendererEmphasis{type} of artefact. To address this, \markdownRendererCite{1}+{}{}{doi:10.1016/j.future.2011.08.004} introduced the notion of \markdownRendererStrongEmphasis{research objects} – \markdownRendererEmphasis{semantically rich aggregations of (potentially distributed) resources that provide a layer of structure on top of information delivered in a machine-readable format}. A Research Object combines the ability to bundle multiple types of artefacts together, such as CSV files, code, examples, and figures. This provides a compelling vision as an approach for implementing FAIR. However, existing research object implementations require a large technology stack, are tailored to a particular platform and are also not easily usable by end-users. \markdownRendererInterblockSeparator
{}To address this gap, a new community came together \markdownRendererCite{1}+{}{}{doi:10.5281/zenodo.3250687} to develop \markdownRendererStrongEmphasis{RO-Crate} - an \markdownRendererEmphasis{approach to package and aggregate research artefacts with their metadata and relationships}. The aim of this paper is to introduce RO-Crate and assess it as a strategy for making multiple types of research artefacts FAIR. Specifically, the contributions of this paper are as follows:\markdownRendererInterblockSeparator
{}\markdownRendererOlBeginTight
\markdownRendererOlItemWithNumber{1}an introduction to RO-Crate, its purpose and context;\markdownRendererOlItemEnd 
\markdownRendererOlItemWithNumber{2}a guide to the RO-Crate community and tooling;\markdownRendererOlItemEnd 
\markdownRendererOlItemWithNumber{3}and an exemplar usage of RO-Crate for different artefacts in different communities as well as its use as a connective tissue for such artefacts.\markdownRendererOlItemEnd 
\markdownRendererOlEndTight \markdownRendererInterblockSeparator
{}The rest of this paper is organised as follows. We first describe RO-Crate, the assumptions underlying it, and define RO-Crate technically and formally. We then proceed to introduce the community and tooling. We move to analyse RO-Crate with respect to usage in a diverse set of domains. Finally, we present related work and conclude with some remarks including RO-Crate highlights and future work. \relax